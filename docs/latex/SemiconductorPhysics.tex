%% Generated by Sphinx.
\def\sphinxdocclass{report}
\documentclass[a4paper,10pt,english]{sphinxmanual}
\ifdefined\pdfpxdimen
   \let\sphinxpxdimen\pdfpxdimen\else\newdimen\sphinxpxdimen
\fi \sphinxpxdimen=.75bp\relax
\ifdefined\pdfimageresolution
    \pdfimageresolution= \numexpr \dimexpr1in\relax/\sphinxpxdimen\relax
\fi
%% let collapsible pdf bookmarks panel have high depth per default
\PassOptionsToPackage{bookmarksdepth=5}{hyperref}

\PassOptionsToPackage{booktabs}{sphinx}
\PassOptionsToPackage{colorrows}{sphinx}

\PassOptionsToPackage{warn}{textcomp}
\usepackage[utf8]{inputenc}
\ifdefined\DeclareUnicodeCharacter
% support both utf8 and utf8x syntaxes
  \ifdefined\DeclareUnicodeCharacterAsOptional
    \def\sphinxDUC#1{\DeclareUnicodeCharacter{"#1}}
  \else
    \let\sphinxDUC\DeclareUnicodeCharacter
  \fi
  \sphinxDUC{00A0}{\nobreakspace}
  \sphinxDUC{2500}{\sphinxunichar{2500}}
  \sphinxDUC{2502}{\sphinxunichar{2502}}
  \sphinxDUC{2514}{\sphinxunichar{2514}}
  \sphinxDUC{251C}{\sphinxunichar{251C}}
  \sphinxDUC{2572}{\textbackslash}
\fi
\usepackage{cmap}
\usepackage[T1]{fontenc}
\usepackage{amsmath,amssymb,amstext}
\usepackage{babel}



\usepackage{tgtermes}
\usepackage{tgheros}
\renewcommand{\ttdefault}{txtt}



\usepackage[Bjarne]{fncychap}
\usepackage{sphinx}

\fvset{fontsize=auto}
\usepackage{geometry}


% Include hyperref last.
\usepackage{hyperref}
% Fix anchor placement for figures with captions.
\usepackage{hypcap}% it must be loaded after hyperref.
% Set up styles of URL: it should be placed after hyperref.
\urlstyle{same}


\usepackage{sphinxmessages}



    \usepackage{amsmath}
    \usepackage{amssymb}
    \usepackage{graphicx}
    

\title{Semiconductor Physics Documentation}
\date{Dec 11, 2024}
\release{}
\author{Y.G}
\newcommand{\sphinxlogo}{\vbox{}}
\renewcommand{\releasename}{}
\makeindex
\begin{document}

\ifdefined\shorthandoff
  \ifnum\catcode`\=\string=\active\shorthandoff{=}\fi
  \ifnum\catcode`\"=\active\shorthandoff{"}\fi
\fi

\pagestyle{empty}
\sphinxmaketitle
\pagestyle{plain}
\sphinxtableofcontents
\pagestyle{normal}
\phantomsection\label{\detokenize{index::doc}}



\chapter{Index}
\label{\detokenize{index:index}}
\sphinxstepscope


\section{Basic physics of semiconductors}
\label{\detokenize{Basic_physics_of_semiconductors:basic-physics-of-semiconductors}}\label{\detokenize{Basic_physics_of_semiconductors::doc}}
\sphinxstepscope


\subsection{Strain calculation}
\label{\detokenize{strain_calculation:strain-calculation}}\label{\detokenize{strain_calculation::doc}}
\sphinxstepscope


\subsection{Band theory}
\label{\detokenize{band_theory:band-theory}}\label{\detokenize{band_theory::doc}}
\sphinxstepscope


\section{Basic equations}
\label{\detokenize{Basic_equation:basic-equations}}\label{\detokenize{Basic_equation::doc}}
\sphinxAtStartPar
In this section, we introduce some basic equations related to semiconductor devices.

\begin{sphinxcontents}
\sphinxstylecontentstitle{Index}
\begin{itemize}
\item {} 
\sphinxAtStartPar
\phantomsection\label{\detokenize{Basic_equation:id1}}{\hyperref[\detokenize{Basic_equation:basic-equations}]{\sphinxcrossref{Basic equations}}}
\begin{itemize}
\item {} 
\sphinxAtStartPar
\phantomsection\label{\detokenize{Basic_equation:id2}}{\hyperref[\detokenize{Basic_equation:poisson-equation}]{\sphinxcrossref{Poisson equation}}}

\item {} 
\sphinxAtStartPar
\phantomsection\label{\detokenize{Basic_equation:id3}}{\hyperref[\detokenize{Basic_equation:current-density-equations}]{\sphinxcrossref{Current\sphinxhyphen{}density equations}}}

\item {} 
\sphinxAtStartPar
\phantomsection\label{\detokenize{Basic_equation:id4}}{\hyperref[\detokenize{Basic_equation:continuity-equations}]{\sphinxcrossref{Continuity equations}}}

\end{itemize}

\end{itemize}
\end{sphinxcontents}


\subsection{Poisson equation}
\label{\detokenize{Basic_equation:poisson-equation}}
\sphinxAtStartPar
The electrostatic potential can be calculated with the corresponding charge distribution \(\rho\) with Poisson equation.
\begin{equation*}
\begin{split}\nabla \cdot \left(\varepsilon_s \nabla\Psi\right) = -\rho,\tag{1}\end{split}
\end{equation*}
\sphinxAtStartPar
where \(\varepsilon_s\) is the dielectric permittivity and \(\varepsilon_s = 11.9 \varepsilon_0\) for Si.
\(\Psi\) is the electrostatic potential.
The electric charge density in a semiconductor is given by the summation of the electron charge density \(n\), the hole charge density \(p\), and the ionized impurity doping density \(D\).
Therefore,
\begin{equation*}
\begin{split}\rho = q(n - p + D),\tag{2}\end{split}
\end{equation*}
\sphinxAtStartPar
where \(q\) is the elementary charge.
Note that \(D\) consists of the ionized acceptor and donor type impurity densities, which mean \(D = N_A - N_D\).

\sphinxAtStartPar
Thus, Eq. 1 can be expressed as following,
\begin{equation*}
\begin{split}\nabla^2\Psi = -\frac{q(n - p + N_A - N_D)}{\varepsilon_s}\tag{3}.\end{split}
\end{equation*}
\sphinxAtStartPar
The left side can be rewritten in the orthogonal coordinate system,
\begin{equation*}
\begin{split}\nabla^2\Psi(x, y, z) = \frac{\partial^2\Psi}{\partial x^2} + \frac{\partial^2\Psi}{\partial y^2} + \frac{\partial^2\Psi}{\partial z^2}\tag{4}.\end{split}
\end{equation*}
\sphinxAtStartPar
For a 1D problem, Eq. 4 can be reduced to
\begin{equation*}
\begin{split}\frac{d^2\Psi_i}{d x^2} = -\frac{d\xi}{dx} = -\frac{\rho}{\varepsilon_s} =  -\frac{q(n - p + N_A - N_D)}{\varepsilon_s},\tag{5}\end{split}
\end{equation*}
\sphinxAtStartPar
Of course, \(\xi = - \nabla\Psi\) holds in Eq. 5.

\sphinxAtStartPar
Poisson equation is often used to determine the distributions of electrostatic potential and electric field caused by a charge density \(\rho\).


\subsection{Current\sphinxhyphen{}density equations}
\label{\detokenize{Basic_equation:current-density-equations}}
\sphinxAtStartPar
The common current equation consists of the drift component, caused by the electric field, and the diffusion component, caused by the gradient of the carrier concentration.
The current density equations are below,
\begin{equation*}
\begin{split}\mathbf{J_n} = q\mu_nn\xi + qD_n\nabla n,\tag{6}\end{split}
\end{equation*}\begin{equation*}
\begin{split}\mathbf{J_p} = q\mu_pp\xi - qD_p\nabla p,\tag{7}\end{split}
\end{equation*}
\sphinxAtStartPar
and
\begin{equation*}
\begin{split}\mathbf{J_{conduction}} = \mathbf{J_n} + \mathbf{J_p},\tag{8}\end{split}
\end{equation*}
\sphinxAtStartPar
where \(\mathbf{J_n}\) and \(\mathbf{J_p}\) are the electron and hole current densities, respectively.
\(\mu_n\) and \(\mu_ p\) are the electron and hole mobilities.
For nondegenerate semiconductors, the carrier diffusion constants (\(D_n\) and \(D_p\)) and the mobilities are given by the Einstein relation,
\begin{equation*}
\begin{split}D_n = \frac{kT}{q}\mu_n,\tag{9}\end{split}
\end{equation*}\begin{equation*}
\begin{split}D_p = \frac{kT}{q}\mu_p.\tag{10}\end{split}
\end{equation*}
\sphinxAtStartPar
Therefore, for a 1D case, Eqs. 6 and 7 can be reduced to
\begin{equation*}
\begin{split}J_n = q\mu_nn\xi + qD_n\frac{dn}{dx} = q\mu_n\left(n\xi + \frac{kT}{q}\frac{dn}{dx}\right) = \mu_nn\frac{dE_{Fn}}{dx},\tag{11}\end{split}
\end{equation*}
\sphinxAtStartPar
and
\begin{equation*}
\begin{split}J_p = q\mu_pp\xi - qD_p\frac{dp}{dx} = q\mu_p\left(p\xi - \frac{kT}{q}\frac{dp}{dx}\right) = \mu_pp\frac{dE_{Fp}}{dx},\tag{12}\end{split}
\end{equation*}
\sphinxAtStartPar
where \(E_{Fn}\) and \(E_{Fp}\) are quasi Fermi levels for electrons and holes, respectively.

\sphinxAtStartPar
Note that these equations are valid for low electric field \(\xi\).
If the electric field is sufficiently high, the term \(\mu_n\xi\) or \(\mu_p\xi\) should be replaced by the saturation velocity \(v_s\).
The last equalities about \(E_{Fn}\) and \(E_{Fp}\) do not hold any more either.


\subsection{Continuity equations}
\label{\detokenize{Basic_equation:continuity-equations}}
\sphinxAtStartPar
While the above current\sphinxhyphen{}density equations hold for steady\sphinxhyphen{}state conditions, the continuity equations deal with time\sphinxhyphen{}dependent states such as low\sphinxhyphen{}level injection, generation, and recombination.

\sphinxstepscope


\section{MIS structure}
\label{\detokenize{MIS_structure:mis-structure}}\label{\detokenize{MIS_structure::doc}}
\sphinxAtStartPar
This document is in progress

\sphinxstepscope


\section{Bipolar transistor}
\label{\detokenize{Bipolar_transistor:bipolar-transistor}}\label{\detokenize{Bipolar_transistor::doc}}
\sphinxAtStartPar
This document is in progress

\sphinxstepscope


\section{MOSFET}
\label{\detokenize{MOSFET:mosfet}}\label{\detokenize{MOSFET::doc}}
\sphinxAtStartPar
This document is in progress

\sphinxAtStartPar
This documentation explains semiconuctor physics.
There will be updates soon!!
Stay tuned.

\sphinxAtStartPar
\sphinxcode{\sphinxupquote{download}}



\renewcommand{\indexname}{Index}
\printindex
\end{document}